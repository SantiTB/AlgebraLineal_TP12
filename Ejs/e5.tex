\item Sean $V$ y $W$ dos $\K$-EV tales que $dim(V)=n$ y $dim(W)=m$. Consideremos las bases $B_V=\{v_1,\dots,v_n\}$ y $B_W=\{w_1,\dots,w_m\}$ de $V$ y $W$, respectivamente. Si $A=[T]_{B_V}$ para $T\in L(V)$ y $C=[S]_{B_W}$ para $S\in L(W)$.
    \begin{enumerate}
        \item Probar que
            \begin{center}
                $[T\otimes S]_{B_V\otimes B_W}=\begin{pmatrix}
                    a_{11}C&a_{12}C&\dots&a_{1n}C\\
                    a_{21}C&a_{22}C&\dots&a_{2n}C\\
                    \vdots &\vdots &\ddots&\vdots\\
                    a_{n1}C&a_{n2}C&\dots&a_{nn}C
                \end{pmatrix}\in\K^{nm\times nm}.$
            \end{center}
            \begin{mdframed}[style=s]
                
            \end{mdframed}
        \item Probar que $tr(T\otimes S)=tr(A)tr(C)$ y $det(T\otimes S)=det(A)^mdet(C)^n$.
            \begin{mdframed}[style=s]
                
            \end{mdframed}
        \item Si $\lambda_1,\dots,\lambda_n$ son los autovalores de $T$ y $\mu_1,\dots,\mu_m$ son los autovalores de $S$ (contando multiplicidades), entonces los $nm$ autovalores de $T\otimes S$ son de la forma\[\eta_{ij}=\lambda_i\mu_j\quad\text{para }i=1,\dots,n;j=1,\dots,m.\]
            Deducir el ítem $b$ a partir de esto.
            \begin{mdframed}[style=s]
                
            \end{mdframed}
    \end{enumerate}